\chapter{Dataset and Snapshot Layer}\label{chap:dsl}

The \emph{Dataset and Snapshot Layer}, or \emph{DSL}  provides a mechanism
for describing and managing relationships-between and properties-of object sets.
Before describing the DSL and the relationships it describes,
a brief overview of the various flavors of object sets is necessary.

\subsection{Object Set Overview}

ZFS provides the ability to create four kinds of object sets:
\emph{filesystems}, \emph{clones}, \emph{snapshots}, and \emph{volumes}.

\begin{itemize}
  \item{ZFS filesystems:}
  A filesystem stores and organizes objects in an easily accessible, POSIX compliant manner.
  \item{ZFS clone:}
  A clone is identical to a filesystem with the exception of its origin.
  Clones originate from snapshots and their initial contents are identical to
  that of the snapshot from which it originated.
  \item{ZFS snapshot:}
  A snapshot is a read-only version of a filesystem, clone,
  or volume at a particular point in time.
  \item{ZFS volume:}
  A volume is a logical volume that is exported by ZFS as a block device.
\end{itemize}

ZFS supports several operations and/or configurations
which cause interdependencies amongst object sets.
The purpose of the DSL is to manage these relationships.
The following is a list of such relationships.

\begin{itemize}
\item{Clones:}
  A clone is related to the snapshot from which it originated.
  Once a clone is created,
  the snapshot in which it originated can not be deleted unless the clone is also deleted.
\item{Snapshots:}
  A snapshot is a point-in-time image of the data in the object set in which it was created.
  A filesystem, clone, or volume can not be destroyed unless its snapshots are also destroyed.
\item{Children:}
  ZFS support hierarchically structured object sets;
  object sets within object sets.
  A child is dependent on the existence of its parent.
  A parent can not be destroyed without first destroying all children.
\end{itemize}

\section{DSL Infrastructure}

Each object set is represented in the DSL as a dataset.
A dataset manages space consumption statistics for an object set,
contains object set location information,
and keeps track of any snapshots inter-dependencies.

Datasets are grouped together hierarchically into collections called Dataset Directories.
Dataset Directories manage a related grouping of datasets and the properties
associated with that grouping.
A DSL directory always has exactly one “active dataset”.
All other datasets under the DSL directory are related to the “active” dataset
through snapshots, clones, or child/parent dependencies.

\begin{figure}[ht]
  \centering
  \includegraphics[width=\textwidth]{Figures/zfs_dsl_infra.pdf}
  \caption{DSL Infrastructure}
  \label{fig:dsl_infra}
\end{figure}

The Illustration~\ref{fig:dsl_infra} shows the DSL infrastructure
including a pictorial view of
how object set relationships are described via the DSL datasets and DSL directories.
The top level DSL Directory can be seen at the top/center of this figure.
Directly below the DSL Directory is the “active dataset”.
The active dataset represents the live  filesystem.
Originating from the active dataset is a linked list of snapshots
which have been taken at different points in time.
Each dataset structure points to a DMU Object Set
which is the actual object set containing object data.
To the left of the top level DSL Directory is a child ZAP object
containing a listing of all child/parent dependencies.
To the right of the DSL directory is a properties ZAP object
containing properties for the datasets within this DSL directory.
A listing of all properties can be seen in Table~\ref{tbl:dd_property} below.

A detailed description of Datasets and DSL Directories are described
in the \nameref{sec:ds_internals}
(section \ref{sec:ds_internals})
and \nameref{sec:dsl_dir_internals}
(section \ref{sec:dsl_dir_internals})
sections below.

\section{DSL Implementation Details}\label{sec:dsl_details}

The DSL is implemented as an object set of type \lstinline{DMU_OST_META}.
This object set is often called the \emph{Meta Object Set}, or \emph{MOS}.
There is only one MOS per pool and the uberblock (see Chapter~\ref{chap:vdev}) points to it directly.

There is a single distinguished object in the Meta Object Set.
This object is called the object directory
and is always located in the second element of the dnode array (index 1).
All objects, with the exception of the object directory,
can be located by traversing through a set of object references starting at this object.

\subsection{The object directory}

The object directory is a ZAP object (an object containing name/value pairs --
see Chapter~\ref{chap:zap} for a description of ZAP objects)
containing three attribute pairs (name/value) named:
root\_dataset, config, and sync\_bplist.

\begin{description}
\item[root\_dataset:]
  The ``root\_dataset''  attribute contains a 64 bit integer value
  identifying the object number of the root DSL directory for the pool.
  The root DSL directory is a special object
  whose contents reference all top level datasets within the pool.
  The ``root\_dataset'' directory,
  is an object of type \lstinline{DMU_OT_DSL_DIR}
  and will be explained in greater detail in Section~\ref{sec:dsl_dir_internals}: \nameref{sec:dsl_dir_internals}

\item[config:]
  The “config” attribute contains a 64 bit integer value
  identifying the object number for an object of type \lstinline{DMU_OT_PACKED_NVLIST}.
  This object contains XDR\_EN-CODED name value pairs
  describing this pools vdev configuration.
  Its contents are similar to those described in Section~\ref{subsec:nvlist}: \nameref{subsec:nvlist}.

\item[sync\_bplist:]
  The ``sync\_bplist'' attribute contains a 64 bit integer value
  identifying the object number for an object of type \lstinline{DMU_OT_SYNC_BPLIST}.
  This object contains a list of block pointers which need to be freed during the next transaction.
\end{description}

The illustration below shows the meta object set (MOS)
in relation to the uberblock and label structures discussed in Chapter~\ref{chap:vdev}.

\begin{figure}[ht]
  \centering
  \includegraphics[width=\textwidth]{Figures/zfs_ondisk_mos.pdf}
  \caption{Meta Object Set}
  \label{fig:mos}
\end{figure}

\section{Dataset Internals}\label{sec:ds_internals}

Datasets are stored as an object of type \lstinline{DMU_OT_DSL_DATASET}.
This object type uses the bonus buffer in the \lstinline{dnode_phys_t}
to hold a \lstinline{dsl_dataset_phys_t} structure.
The contents of the \lstinline{dsl_dataset_phys_t} structure are shown below.

\begin{description}
\item[uint64\_t ds\_dir\_obj]
  Object number of the DSL directory referencing this dataset.

\item[uint64\_t ds\_prev\_snap\_obj]
  If this dataset represents a filesystem, volume, or clone,
  this field contains the 64 bit object number for the most recent snapshot taken;
  this field is zero if no snapshots have been taken.

  If this dataset represents a snapshot,
  this field contains the 64 bit object number for the snapshot taken prior to this snapshot.
  This field is zero if there are no previous snapshots.

\item[uint64\_t ds\_prev\_snap\_txg]
  The transaction group number when the previous snapshot
  (pointed to by \lstinline{ds_prev_snap_obj}) was taken.

\item[uint64\_t ds\_next\_snap\_ob]
  This field is only used for datasets representing snapshots.
  It contains the object number of the dataset
  which is the most recent snapshot.
  This field is always zero for datasets representing clones, volumes, or filesystems.

\item[uint64\_t ds\_snapnames\_zapobj]
  Object number of a ZAP object (see Chapter~\ref{chap:zap})
  containing name value pairs for each snapshot of this dataset.
  Each pair contains the name of the snapshot
  and the object number associated with it's DSL dataset structure.

\item[uint64\_t ds\_num\_children]
  Always zero if not a snapshot.
  For snapshots, this is the number of references to this snapshot: 1
  (from the next snapshot taken, or from the active dataset if no snapshots have been taken)
  + the number of clones originating from this snapshot.

\item[uint64\_t ds\_creation\_time]
  Seconds since January $1^{st}$ 1970 (GMT) when this dataset was created.

\item[uint64\_t ds\_creation\_txg]
  The transaction group number in which this dataset  was created.

\item[uint64\_t ds\_deadlist\_obj]
  The object number of the deadlist (an array of blkptr's deleted since the last snapshot).

\item[uint64\_t ds\_used\_bytes]
  unique bytes used by the object set represented by this dataset

\item[uint64\_t ds\_compressed\_bytes]
  number of compressed bytes in the object set represented by this dataset

\item[uint64\_t ds\_uncompressed\_bytes]
  number of uncompressed bytes in the object set represented by this dataset

\item[uint64\_t ds\_unique\_bytes]
  When a snapshot is taken,
  its initial contents are identical to that of the active copy of the data.
  As the data changes in the active copy,
  more and more data becomes unique to the snapshot
  (the data diverges from the snapshot).
  As that happens, the amount of data unique to the snapshot increases.
  The amount of unique snapshot data is stored in this field
  it is zero for clones, volumes, and filesystems.

\item[uint64\_t ds\_fsid\_guid]
  64 bit ID that is guaranteed to be unique amongst all currently open datasets.
  Note, this ID could change between successive dataset opens.

\item[uint64\_t ds\_guid]
  64 bit global id for this dataset.
  This value never changes during the lifetime of the object set.

\item[uint64\_t ds\_restoring]
  The field is set to ``1'' if ZFS is in the process of
  restoring to this dataset through ``'zfs restore''
  \verbfootnote{See the ZFS Admin Guide for information about the zfs command.}

\item[blkptr\_t ds\_bp]
  Block pointer containing the location of the object set that this dataset represents.
\end{description}

\section{DSL Directory Internals}\label{sec:dsl_dir_internals}

The DSL Directory object contains a \lstinline{dsl_dir_phys_t} structure in its bonus buffer.
The contents of this structure are described in detail below.

\begin{description}
\item[uint64\_t dd\_creation\_time]
  Seconds since January 1st, 1970 (GMT) when this DSL directory was created.

\item[uint64\_t dd\_head\_dataset\_obj]
  64 bit object number of the active dataset object

\item[uint64\_t dd\_parent\_obj]
  64 bit object number of the parent DSL directory

\item[uint64\_t dd\_clone\_parent\_obj]
  For cloned object sets, this field contains the object number of snapshot used to create this clone.

\item[uint64\_t dd\_child\_dir\_zapobj]
  Object number of a ZAP object containing name-value pairs for each child of this DSL directory.

\item[uint64\_t dd\_used\_bytes]
  Number of bytes used by all datasets within this directory:
  includes any snapshot and child dataset used bytes.

\item[uint64\_t dd\_compressed\_bytes]
  Number of compressed bytes for all datasets within this DSL directory.

\item[uint64\_t dd\_uncompressed\_bytes]
  Number of uncompressed bytes for all datasets within this DSL directory.

\item[uint64\_t dd\_quota]
  Designated quota, if any, which can not be exceeded by the datasets within this DSL directory.

\item[uint64\_t dd\_reserved]
  The amount of space reserved for consumption by the datasets within this DSL directory.

\item[uint64\_t dd\_props\_zapobj]
  64 bit object number of a ZAP object containing the properties
  for all datasets within this DSL directory.
  Only the non-inherited/locally set values are represented in this ZAP object.
  Default, inherited values are inferred when there is an absence of an entry.
\end{description}

\begin{LongTable3Columns}{Property}{Description}{Values}
  {Editable Property Values stored in the dd\_props\_zabobj}{dd_property}
  {lp{.45\textwidth}p{.3\textwidth}}
  {
    aclinherit
    & Controls inheritance behavior for datasets.
    & \begin{minipage}[t]{.3\textwidth}
        \begin{itemize}[label={}, labelsep=0pt, leftmargin=0pt, noitemsep]
        \item discard = 0
        \item noallow = 1
        \item passthrough = 3
        \item secure = 4 (default)
        \end{itemize}
      \end{minipage}
    \rule[-.5ex]{0pt}{0pt}\\
    aclmode
    & Controls chmod and file/dir creation behavior for datasets.
    & \begin{minipage}[t]{.3\textwidth}
        \begin{itemize}[label={}, labelsep=0pt, leftmargin=0pt, noitemsep]
        \item discard = 0
        \item groupmask = 2 (default)
        \item passthrough = 3
        \end{itemize}
      \end{minipage}
    \rule[-.5ex]{0pt}{0pt}\\
    atime
    & Controls whether atime is updated on objects within a dataset.
    & \begin{minipage}[t]{.3\textwidth}
        \begin{itemize}[label={}, labelsep=0pt, leftmargin=0pt, noitemsep]
        \item off = 0
        \item on = 1 (default)
        \end{itemize}
      \end{minipage}
    \rule[-.5ex]{0pt}{0pt}\\
    checksum
    & Checksum algorithm for all datasets within this DSL Directory.
    & \begin{minipage}[t]{.3\textwidth}
        \begin{itemize}[label={}, labelsep=0pt, leftmargin=0pt, noitemsep]
        \item off = 0
        \item on = 1 (default)
        \end{itemize}
      \end{minipage}
    \rule[-.5ex]{0pt}{0pt}\\
    compression
    & Compression algorithm for all datasets within this DSL Directory.
    & \begin{minipage}[t]{.3\textwidth}
        \begin{itemize}[label={}, labelsep=0pt, leftmargin=0pt, noitemsep]
        \item off = 0 (default)
        \item on = 1
        \end{itemize}
      \end{minipage}
    \rule[-.5ex]{0pt}{0pt}\\
    devices
    & Controls whether device nodes can be opened on datasets.
    & \begin{minipage}[t]{.3\textwidth}
        \begin{itemize}[label={}, labelsep=0pt, leftmargin=0pt, noitemsep]
        \item devices = 0
        \item nodevices = 1 (default)
        \end{itemize}
      \end{minipage}
    \rule[-.5ex]{0pt}{0pt}\\
    exec
    & Controls whether files can be executed on a dataset.
    & \begin{minipage}[t]{.3\textwidth}
        \begin{itemize}[label={}, labelsep=0pt, leftmargin=0pt, noitemsep]
        \item noexec = 0
        \item exec = 1 (default)
        \end{itemize}
      \end{minipage}
    \rule[-.5ex]{0pt}{0pt}\\
    mountpoint
    & Mountpoint path for datasets within this DSL Directory.
    & \begin{minipage}[t]{.3\textwidth}
        string
      \end{minipage}
    \rule[-.5ex]{0pt}{0pt}\\
    quota
    & Limits the amount of space all datasets within a DSL directory can consume.
    & quota size in bytes or zero for no quota (default)
    \rule[-.5ex]{0pt}{0pt}\\
    readonly
    & Controls whether objects can be modified on a dataset.
    & \begin{minipage}[t]{.3\textwidth}
        \begin{itemize}[label={}, labelsep=0pt, leftmargin=0pt, noitemsep]
        \item readwrite = 0 (default)
        \item readonly  = 1
        \end{itemize}
      \end{minipage}
    \rule[-.5ex]{0pt}{0pt}\\
    recordsize
    & Block Size for all objects within the datasets contained in this DSL Directory
    & record size in bytes
    \rule[-.5ex]{0pt}{0pt}\\
    reservation
    & Amount of space reserved for this DSL Directory, including all child datasets and child DSL Directories.
    & reservation size in bytes
    \rule[-.5ex]{0pt}{0pt}\\
    setuid
    & Controls whether the set-UID bit is respected on a dataset.
    & \begin{minipage}[t]{.3\textwidth}
        \begin{itemize}[label={}, labelsep=0pt, leftmargin=0pt, noitemsep]
        \item nosetuid  = 0
        \item setuid = 1 (default)
        \end{itemize}
      \end{minipage}
    \rule[-.5ex]{0pt}{0pt}\\
    sharenfs
    & Controls whether  the datasets in a DSL Directory  are shared by NFS.
    & string -- any valid nfs share options
    \rule[-.5ex]{0pt}{0pt}\\
    snapdir
    & Controls whether \lstinline{.zfs} is hidden or visible in the root filesystem.
    & \begin{minipage}[t]{.3\textwidth}
        \begin{itemize}[label={}, labelsep=0pt, leftmargin=0pt, noitemsep]
        \item hidden  = 0
        \item visible = 1 (default)
        \end{itemize}
      \end{minipage}
    \rule[-.5ex]{0pt}{0pt}\\
    volblocksize
    & For volumes, specifies the block size of the volume.
      The blocksize cannot be changed once the volume has been written,
      so it should be set at volume creation time
    & between 512 to 128K, powers of two. Defaults to 8K.
    \rule[-.5ex]{0pt}{0pt}\\
    volsize
    & Volume size, only applicable to volumes.
    & volume size in bytes
    \rule[-.5ex]{0pt}{0pt}\\
    zoned
    & Controls whether a dataset is managed through a local zone.
    & \begin{minipage}[t]{.3\textwidth}
        \begin{itemize}[label={}, labelsep=0pt, leftmargin=0pt, noitemsep]
        \item off  = 0 (default)
        \item on = 1
        \end{itemize}
      \end{minipage}
    \rule[-.5ex]{0pt}{0pt}\\
  }
\end{LongTable3Columns}
