\chapter*{Introduction}\addcontentsline{toc}{chapter}{Introduction}

ZFS is a new filesystem technology
that provides immense capacity (128-bit),
provable data integrity,
always-consistent on-disk format,
self-optimizing performance,
and real-time remote replication.

ZFS departs from traditional filesystems by eliminating the concept of volumes.
Instead,
ZFS filesystems share a common storage pool
consisting of writeable storage media.
Media can be added or removed from the pool
as filesystem capacity requirements change.
Filesystems dynamically grow and shrink as needed
without the need to re-partition underlying storage.

ZFS provides a truly consistent on-disk format,
but using a \emph{copy on write} (\emph{COW}) transaction model.
This model ensures that on disk data is never overwritten
and all on disk updates are done atomically.

The ZFS software is comprised of seven distinct pieces:
the \emph{SPA} (Storage Pool Allocator),
the \emph{DSL} (Dataset and Snapshot Layer),
the \emph{DMU} (Data Management Layer),
the \emph{ZAP} (ZFS Attribute Processor),
the \emph{ZPL} (ZFS Posix Layer),
the \emph{ZIL} (ZFS Intent Log),
and \emph{ZVOL} (ZFS Volume).
The on-disk structures associated with each of these pieces are explained in the following chapters:
SPA ({Chapters \ref{chap:vdev} and \ref{chap:blkptr}}),
DSL (Chapter \ref{chap:dsl}),
DMU (Chapter \ref{chap:dmu}),
ZAP (Chapter \ref{chap:zap}),
ZPL (Chapter \ref{chap:zpl}),
ZIL (Chapter \ref{chap:zil}),
ZVOL (Chapter \ref{chap:zvol}).

