\chapter{ZFS Attribute Processor}\label{chap:zap}

The \emph{ZFS Attribute Processor}, or \emph{ZAP} is a module
which sits on top of the DMU and operates on objects called ZAP objects.
A ZAP object is a DMU object used to store attributes in the form of name-value pairs.
The name portion of the attribute is a zero-terminated string of up to 256 bytes
(including terminating NULL).
The value portion of the attribute is an array of integers
whose size is only limited by the size of a ZAP data block.

ZAP objects are used to store properties for a dataset,
navigate filesystem objects, store pool properties and more.
The following table contains a list of ZAP object types.

\begin{table}[ht]
  \centering
  \caption{ZAP Object Types}\label{tbl:zap_obj_types}
  \begin{tabular}{l}
    \toprule
    \multicolumn{1}{c}{\textbf{ZAP Object Type}}\\
    \midrule
    \lstinline!DMU_OT_OBJECT_DIRECTORY!\\
    \lstinline!DMU_OT_DSL_DIR_CHILD_MAP!\\
    \lstinline!DMU_OT_DSL_DS_SNAP_MAP!\\
    \lstinline!DMU_OT_DSL_PROPS!\\
    \lstinline!DMU_OT_DIRECTORY_CONTENTS!\\
    \lstinline!DMU_OT_MASTER_NODE!\\
    \lstinline!DMU_OT_DELETE_QUEUE!\\
    \lstinline!DMU_OT_ZVOL_PROP!\\
    \bottomrule
  \end{tabular}
\end{table}

ZAP objects come in two forms; \emph{microzap} objects and \emph{fatzap} objects.
Microzap objects are a lightweight version of the fatzap
and provide a simple and fast lookup mechanism for a small number of attribute entries.
The fatzap is better suited for ZAP objects containing large numbers of attributes.

The following guidelines are used by ZFS to decide
whether or not to use a fatzap or a microzap object.
\begin{center}
\begin{minipage}[c]{.8\textwidth}
  A microzap object is used if all three conditions below are met:
  \begin{itemize}
  \item all name-value pair entries fit into one block.
    The maximum data block size in ZFS is 128KB and
    this size block can fit up to 2047 microzap entries.
  \item The value portion of all attributes are of type \lstinline{uint64_t}.
  \item The name portion of each attribute is less than or equal to 50 characters in length
    (including NULL terminating character).
  \end{itemize}
  If any of the above conditions are not met, a fatzap object is used.
\end{minipage}
\end{center}

The first 64 bit word in each block of a ZAP object
is used to identify the type of ZAP contents contained within this block.
The table below shows these values.

\begin{table}[ht]
  \centering
  \caption{ZAP Object Block Types}\label{tbl:zap_obj_block_types}
  \begin{tabular}{lp{5cm}l}
    \toprule
    \multicolumn{1}{c}{\textbf{Identifier}} &
    \multicolumn{1}{c}{\textbf{Description}} &
    \multicolumn{1}{c}{\textbf{Value}}\\
    \midrule
    ZBT\_MICRO & This block contains microzap entries &(1ULL << 63) + 3\\
    ZBT\_HEADER & This block is used for the fatzap.
    This identifier is only used for the first block in a fatzap object.
    & (1ULL << 63) + 1\\
    ZBT\_LEAF & This block is used for the fatzap.
    This identifier is used for all blocks in the fatzap with the exception of the first.
    & (1ULL << 63) + 0\\
    \bottomrule
  \end{tabular}
\end{table}

\section{The Micro Zap}

The microzap implements a simple mechanism for storing a small number of attributes.
A microzap object consists of a single block
containing an array of microzap entries (\lstinline{mzap_ent_phys_t} structures).
Each attribute stored in a microzap object is represented
by one of these microzap entry structures.

\begin{figure}[ht]
  \centering
  \includegraphics[scale=1]{Figures/zfs_mzap.pdf}
  \caption{Microzap Block Layout}
  \label{fig:mzap}
\end{figure}

A microzap block is laid out as follows:
the first 128 bytes of the block contain
a microzap header structure called the \lstinline{mzap_phys_t}.
This structure contains a 64 bit \lstinline{ZBT_MICRO} value
indicating that this block is used to store microzap entries.
Following this value is a 64 bit salt value
that is stirred into the hash so that the hash function is different for each ZAP object.
The next 42 bytes of this header is intentionally left blank
and the last 64 bytes contain the first microzap entry
(a structure of type \lstinline{mzap_ent_phys_t}).
The remaining bytes in this block are used to
store an array of \lstinline{mzap_ent_phys_t} structures.
The Illustration~\ref{fig:mzap} above shows the layout of this block.

The \lstinline{mzap_ent_phys_t} structure and associated \#defines are shown below.

\begin{lstlisting}[style=c]
#define MZAP_ENT_LEN            64
#define MZAP_NAME_LEN           (MZAP_ENT_LEN - 8 – 4 - 2)
typedef struct mzap_ent_phys {
        uint64_t mze_value;
        uint32_t mze_cd;
        uint16_t  mze_pad;
        char mze_name[MZAP_NAME_LEN];
} mzap_ent_phys_t;
\end{lstlisting}

\begin{flushright}
  \begin{minipage}[c]{.92\textwidth}
    \begin{description}[style=standard]
    \item[mze\_value:]  64 bit integer
    \item[mze\_cd:]
      32 bit collision differentiator (``CD''):
      associated with an entry whose hash value is the same as another entry within this ZAP object.
      When an entry is inserted into the ZAP object,
      the lowest CD which is not already used by an entry with the same hash value is assigned.
      In the absence of hash collisions,
      the CD value will be zero.
    \item[mze\_pad:]
      reserved for future use
    \item[mze\_name:]
      NULL terminated string less than or equal to 50 characters in length
    \end{description}
  \end{minipage}
\end{flushright}

\section{The Fat Zap}

The fatzap implements a flexible architecture for storing large numbers of attributes,
and/or attributes with long names or complex values (not \lstinline{uint64_t}).
This section begins with an explanation of the basic structure of a fatzap object
and is followed by a detailed explanation of each component of a fatzap object.

All entries in a fatzap object are arranged based on a 64 bit hash of the attribute's name.
The hash is used to index into a pointer table
(as can be seen on the left side of the illustration below).
The number of bits used to index into this table
(sometimes called the prefix)
is dependent on the number of entries in the table.
The number of entries in the table can change over time.
As policy stands today,
the pointer table will grow if
the number of entries hashing to a particular bucket exceeds
the capacity of one leaf block
(explained in detail below).
The pointer table entries reference a chain of fatzap blocks
called leaf blocks,
represented by the \lstinline{zap_leaf_phys} structure.
Each leaf block is broken up into some number of chunks
(zap\_leaf\_chunks)
and each attribute is stored in one or more of these leaf chunks.
The illustration below shows the basic fatzap structures,
each component is explained in detail in the following sections.

\begin{figure}[ht]
  \centering
  \includegraphics[width=\textwidth]{Figures/zfs_fzap_overview.pdf}
  \caption{Fatzap Structure Overview}
  \label{fig:fzap_overview}
\end{figure}

\subsection{zap\_phys\_t}

\begin{figure}[ht]
  \centering
  \includegraphics[scale=1]{Figures/zfs_zap_phys.pdf}
  \caption{zap\_phys\_t Structure}
  \label{fig:zap_phys_t}
\end{figure}

The first block of a fatzap object contains a 128KB \lstinline{zap_phys_t} structure.
Depending on the size of the pointer table,
this structure may contain the pointer table.
If the pointer table is too large to fit in the space provided by the \lstinline{zap_phys_t},
some information about where it can be found
is stored in the \lstinline{zap_table_phys} portion of this structure.
The definitions of the \lstinline{zap_phys_t} contents are as follows:

\begin{description}

\item[zap\_block\_type:]
64 bit integer identifying type of ZAP block. Always set to ZBT\_HEADER (see Table 14) for the first block in the fatzap.

\item[zap\_magic:]
  64 bit integer containing the ZAP magic number \lstinline{0x2F52AB2AB} (zfs-zap-zap)

\item[zap\_table\_phys:]
  structure whose contents are used to describe the pointer table

  \begin{description}
  \item[zt\_blk:]
    Blkid for the first block of the pointer table.
    This field is only used when the pointer table is external to the \lstinline{zap_phys_t} structure;
    zero otherwise.

  \item[zt\_numblks:]
    Number of blocks used to hold the pointer table.
    This field is only used when the pointer table is external to
    the \lstinline{zap_phys_t} structure; zero otherwise.

  \item[zt\_shift:]
    Number of bits used from the hash value to index into the pointer table.
    If the pointer table is contained within the \lstinline{zap_phys_t},
    this value will be 13.

  \item[uint64\_t zt\_nextblk: uint64\_t zt\_blks\_copied:]
    The above two fields are used when the pointer table changes sizes.
  \end{description}

\item[zap\_freeblk:]
  64 bit integer containing the first available ZAP block that can be used to allocate a new zap\_leaf.

\item[zap\_num\_leafs:]
  Number of zap\_leaf\_phys\_t structures (described below) contained within this ZAP object.

\item[zap\_salt:]
  The salt value is a 64 bit integer that is stirred into the hash function,
  so that the hash function is different for each ZAP object.

\item[zap\_num\_entries:]
  Number of attributes stored in this ZAP object.

\item[zap\_leafs{[8192]}:]
  The zap\_leaf array contains 213 (8192) slots.
  If the pointer table has fewer than 213 entries,
  the pointer table will be stored here.
  If not, this field is unused.
\end{description}

\subsection{Pointer Table}

The pointer table is a hash table which uses a chaining method to handle collisions.
Each hash bucket contains a 64 bit integer which describes
the level zero block id
(see Chapter~\ref{chap:dmu} for a description of block ids)
of the first element in the chain of entries hashed here.
An entries hash bucket is determined by
using the first few bits of the 64 bit ZAP entry hash
computed from the attribute's name.
The value used to index into the pointer table is called
the \emph{prefix} and is the \lstinline{zt_shift} high order bits of the 64 bit computed hash.

\subsection{zap\_leaf\_phys\_t}

The \lstinline{zap_leaf_phys_t} is the structure
referenced by the pointer table.
Collisions in the pointer table result in \lstinline{zap_leaf_phys_t} structures
being strung together in a link list fashion.
The \lstinline{zap_leaf_phys_t} structure contains
a header, a hash table, and some number of chunks.

\begin{lstlisting}[style=c]
typedef struct zap_leaf_phys {
        struct zap_leaf_header {
                uint64_t lh_block_type;
                uint64_t lh_pad1;
                uint64_t lh_prefix;
                uint32_t lh_magic;
                uint16_t lh_nfree;
                uint16_t lh_nentries;
                uint16_t lh_prefix_len;
                uint16_t lh_freelist;
                uint8_t lh_flags;
                uint8_t lh_pad2[11];
        } l_hdr;
        /*
         * The header is followed by a hash table with
         * ZAP_LEAF_HASH_NUMENTRIES(zap) entries.
         */
        uint16_t l_hash[1];
        /*
         * The hash table is followed by an array of
         * ZAP_LEAF_NUMCHUNKS(zap) zap_leaf_chunk structures.
         * These structures are accessed with the
         * ZAP_LEAF_CHUNK() macro.
         */
        union zap_leaf_chunk {
                struct zap_leaf_entry {
                        uint8_t le_type;
                        uint8_t le_value_intlen;
                        uint16_t le_next;
                        uint16_t le_name_chunk;
                        uint16_t le_name_numints;
                        uint16_t le_value_chunk;
                        uint16_t le_value_numints;
                        uint32_t le_cd;
                        uint64_t le_hash;
                } l_entry;
                struct zap_leaf_array {
                        uint8_t la_type;
                        uint8_t la_array[ZAP_LEAF_ARRAY_BYTES];
                        uint16_t la_next;
                } l_array;
                struct zap_leaf_free {
                        uint8_t lf_type;
                        uint8_t lf_pad[ZAP_LEAF_ARRAY_BYTES];
                        uint16_t lf_next;
                } l_free;
        } l_chunk[ZAP_LEAF_NUMCHUNKS];
} zap_leaf_phys_t;
\end{lstlisting}

\subsubsection{Header}
The header for the ZAP leaf is stored in a \lstinline{zap_leaf_header} structure.
It's description is as follows:
\begin{description}

\item[lhr\_block\_type:]
  always \lstinline{ZBT_LEAF}
  (see Table~\ref{tbl:zap_obj_block_types} for values)
\item[lhr\_next:]
  64 bit integer block id for the next leaf in a block chain.
\item[lhr\_prefix \normalfont{and} lhr\_prefix\_len:]
  Each leaf (or chain of leafs) stores the ZAP entries
  whose first \lstinline{lhr_prefixlen} bits of their hash value equals \lstinline{lhr_prefix}.
  \lstinline{lhr_prefixlen} can be equal to or less than \lstinline{zt_shift}
  (the number of bits used to index into the pointer table)
  in which case multiple pointer table buckets reference the same leaf.
\item[lhr\_magic:]
  leaf magic number == \lstinline{0x2AB1EAF} (zap-leaf)
\item[lhr\_nfree:]
  number of free chunks in this leaf (chunks described below)
\item[lhr\_nentries:]
  number of ZAP entries stored in this leaf
\item[lhr\_freelist:]
  head of a list of free chunks,
  16 bit integer used to index into the \lstinline{zap_leaf_chunk} array

\end{description}

\subsubsection{Leaf Hash}
The next 8KB of the \lstinline{zap_leaf_phys_t} is the zap leaf hash table.
The entries in the has table reference chunks of type \lstinline{zap_leaf_entry}.
Twelve bits (the twelve following the \lstinline{lhr_prefix_len}
used to uniquely identify this block)
of the attribute's hash value are used to index into the this table.
Hash table collisions are handled by chaining entries.
Each bucket in the table contains a 16 bit integer
which is the index into the \lstinline{zap_leaf_chunk} array.

\subsection{zap\_leaf\_chunk}

Each leaf contains an array of chunks.
There are three types of chunks:
\lstinline{zap_leaf_entry}, \lstinline{zap_leaf_array}, and \lstinline{zap_leaf_free}.
Each attribute is represented by some number of these chunks:
one \lstinline{zap_leaf_entry} and some number of \lstinline{zap_leaf_array} chunks.
The illustration below shows
how these chunks are arranged.
A detailed description of each chunk type follows the illustration.

\begin{figure}[ht]
  \centering
  \includegraphics[width=\textwidth]{Figures/zfs_zap_leaf.pdf}
  \caption{Zap Leaf Structure}
  \label{fig:zap_leaf}
\end{figure}

\begin{description}
\item[zap\_leaf\_entry:]
  The leaf hash table (described above) points to chucks of this type.
  This entry contains pointers to chunks of type \lstinline{zap_leaf_array}
  which hold the name and value for the attributes being stored here.
  \begin{description}
  \item[le\_type:]
    \lstinline{ZAP_LEAF_ENTRY} == 252
  \item[le\_int\_size:]
    Size of integers in bytes for this entry.
  \item[le\_next:]
    Next entry in the \lstinline{zap_leaf_chunk} chain.
    Chains occur when there are collisions in the hash table.
    The end of the chain is designated by a \lstinline{le_next} value of \lstinline{0xffff}.
  \item[le\_name\_chunk:] 16 bit integer identifying the chunk of type \lstinline{zap_leaf_array}
    which contains the first 21 characters of this attribute's name.
  \item[le\_name\_length:]
    The length of the attribute's name, including the NULL character.
  \item[le\_value\_chunk:]
    16 bit integer identifying the first chunk
    (type \lstinline{zap_leaf_array}) containing the first 21 bytes of the attribute's value.
  \item[le\_value\_length:]
    The length, in integer increments (\lstinline{le_int_size})
  \item[le\_cd:]
    The collision differentiator (``CD'') is a value
    associated with an entry
    whose hash value is the same as another entry within this ZAP object.
    When an entry is inserted into the ZAP object,
    the lowest CD which is not already used by an entry with the same hash value is assigned.
    In the absence of hash collisions, the CD value will be zero.
  \item[le\_hash:]
    64 bit hash of this attribute's name.
  \end{description}

\item[zap\_leaf\_array:]
  Chunks of the \lstinline{zap_leaf_array} hold
  either the name or the value of the ZAP attribute.
  These chunks can be strung together to provide for long names or large values.
  \lstinline{zap_leaf_array} chunks are pointed to by a \lstinline{zap_leaf_entry} chunk.

  \begin{description}
  \item[la\_type:]
    \lstinline{ZAP_LEAF_ARRAY} == 251
  \item[la\_array:]
    21 byte array containing the name or value's value.
    Values of type ``integer'' are always stored in bigendian format,
    regardless of the machine's native endianness.
  \item[la\_next:]
    16 bit integer used to index into the \lstinline{zap_leaf_chunk} array
    and references the next \lstinline{zap_leaf_array} chunk for this attribute;
    a value of \lstinline{0xffff} (\lstinline{CHAIN_END}) is used to designate the end of the chain
  \end{description}

\item[zap\_leaf\_free:]
  Unused chunks are kept in a chained free list.
  The root of the free list is stored in the leaf header.
  \begin{description}
  \item[lf\_type:]
    \lstinline{ZAP_LEAF_FREE} == 253
  \item[lf\_next:]
    16 bit integer pointing to the next free chunk.
  \end{description}
\end{description}
